% !TEX program = xelatex

\documentclass{resume}
\usepackage{zh_CN-Adobefonts_external} % Simplified Chinese Support using external fonts (./fonts/zh_CN-Adobe/)
%\usepackage{zh_CN-Adobefonts_internal} % Simplified Chinese Support using system fonts

\usepackage{fancyhdr}
\pagestyle{fancy}
\renewcommand{\headrulewidth}{0pt}
\renewcommand{\footrulewidth}{0pt}
\fancyfoot[L]{\textit{Compiled by \XeLaTeX{}}}
\fancyfoot[R]{Last Modified: \today}
\fancyfoot[C]{}
\fancyhead{}

\usepackage{ruby}
\renewcommand{\rubysep}{-0.1ex}

\newcommand{\awardline}[3]{\par \hspace{0em}\rlap{\textit{#1}} \hspace{.25\textwidth}\rlap{#2} \hfill #3 \par}

\begin{document}
\pagenumbering{gobble} % suppress displaying page number

\name{\ruby{管}{カン}\,\ruby{清文}{キヨフミ}}

% {E-mail}{mobilephone}{homepage}
% be careful of _ in email address
\contactInfo{i@kiyofumi.me}{(+86) 134 0115 7995}{http://lnked.in/qingwen}
% {E-mail}{mobilephone}
% keep the last empty braces!
%\contactInfo{xxx@yuanbin.me}{(+86) 131-221-87xxx}{}

\section{\faUsers\ 长期工作经历}
\datedsubsection{\textbf{滴滴出行}\quad{}北京}{2016 -- 现在}
\role{研发工程师}{滴滴研究院}
我参与过很多项目,包括预测分单、动态调价、运力调度、供需预测、乘客补贴、热力图等,详见下一页的``项目经历''。 

\datedsubsection{\textbf{微软(Microsoft)}\quad{}北京}{2013 -- 2015 (2年)}
\role{实习生}{微软亚洲研究院}
我参与过很多项目,包括文档校正\,(Document Rectification), 
文字检测\,(Text Detection), 
点击率预估\,(Click Through Rate Prediction), 
\,聊天和跳舞机器人(Talking and Dancing Robot)等, 
详见下一页的``项目经历''。

\section{\faUser\ 短期工作经历}

\datedsubsection{\textbf{谷歌(Google)}\quad{}英国伦敦}{2015年11月 -- 2016年1月}
\role{软件工程师}{Android Google Search App组}
我为Google Now这个安卓应用修复bug和代码重构。

\datedsubsection{\textbf{今日头条}\quad{}北京}{2015年7月 -- 2015年11月}
\role{软件工程师}{国际化组}
我参与从头搭建推荐模块。

\datedsubsection{\textbf{摩根大通银行(J.P. Morgan Chase)}\quad{}北京}{2014年7月 -- 2014年10月}
\role{暑期分析员}{量化研究(Quantitative Research)组}
我对内部使用的库做重构和数据处理。

\section{\faGraduationCap\ 教育经历}
%\datedsubsection{\textbf{Beihang University (BUAA)}, Beijing, China}{2012 -- 2015 (3 yrs)}
%\textit{Master} in Software Engineering.

%\datedsubsection{\textbf{Beihang University (BUAA)}, Beijing, China}{2008 -- 2012 (4 yrs)}
%\textit{Bachelor} in Software Engineering and Applied Mathematics(minor). 
%\begin{itemize}
%  \item High Scored Courses: C++(100), Algorithm(99), Advanced Math(99), Linear Algebra(99), \ldots
%  \item I was recommended as postgraduate candidate exempted from GRE (Graduate Record Examination).
%\end{itemize}

\datedsubsection{\textbf{北京航空航天大学\,(BUAA)}\quad{}北京}{2008 -- 2015 (7年)}
软件工程硕士(保研),软件工程和应用数学学士.
\begin{itemize}
  \item 高分课程: 高级语言程序设计(C++)~(100), 算法(99), 高等数学(99), 线性代数(99), \ldots
\end{itemize}

\section{\faHeartO\ 获奖}
\awardline{第27名}{ACM ICPC World Finals}{2013}
\awardline{第5名,金牌}{ACM ICPC Asia Regionals}{2012}
\awardline{前500名}{Google Code Jam}{2013}
\awardline{前500名}{Facebook Hacker Cup}{2014}
\awardline{前30名}{Microsoft Beauty of Programming National Invition Contest}{2013}
\awardline{2/141}{Tencent Scholarship}{2010}
\awardline{6/141}{Outstanding Student Leader}{2008 -- 2009}
\awardline{12/141}{Outstanding Academic Performance Scholarship}{2008 -- 2009}
  %\item \datedline{Beihang Sanguosha 3v3 Match, \textit{\nth{3} Place}}{2013}

% \section{\faUsers\ Experience}
% \datedsubsection{\textbf{FLAG Inc.} California, America}{2012 -- Present}
% \role{Summer Intern}{Manager: xxx}
% Brief introduction: xxx.
% \begin{itemize}
%   \item Implemented xxx feature
%   \item Optimized xxx 5\%
%   \item xxx
% \end{itemize}

% Reference Test
%\datedsubsection{\textbf{Paper Title\cite{zaharia2012resilient}}}{May. 2015}
%An xxx optimized for xxx\cite{verma2015large}
%\begin{itemize}
%  \item main contribution
%\end{itemize}

\section{\faCogs\ Skills \& Miscellaneous}
\begin{itemize}[parsep=0.5ex]
  \item Languages: C++11 $>$ English $>$ Python2.7
          	$>$ Java $>$ C\# $>$ SQL $>$ Bash
		%$>$ Matlab/Octave
  %\item Skills: Fast-learning, vim, git, *nix
 % \item Hobbies: Zhihu, Bilibili, Sanguosha; Cycling, Yoga
  %Debug with eyes, Vim, Git 
  % \item Platform: Linux
  % \item Development: Web, xxx
\end{itemize}

\section{在滴滴的项目经历}
\datedsubsection{\textbf{预测分单}}{2016-07至今}
 \begin{itemize}
    % \item 在司机和乘客的历史数据中学习接单概率模型,提高司机和乘客的匹配度,利用运力的规模效应实时地从全局上最优化总体交通运输效率和乘客出行体验。
    \item 我们在乘客发单之前,预测出这个潜在订单的应答率;目前结果用在快车和专车的动态调价。
 \end{itemize}

\datedsubsection{\textbf{专车动态调价}}{2016-08至今}
 \begin{itemize}
    \item 我做过项目的负责人。
 \end{itemize}

\datedsubsection{\textbf{专车乘客补贴}}{2016-07至今}
 \begin{itemize}
    \item 我从头搭建了专车的补贴框架,含回放功能,500行Google-styled C++11代码,thrift服务,robust,负载均衡。
 \end{itemize}

\datedsubsection{\textbf{运力调度}(负责人)}{2016-02至2016-07}
 \begin{itemize}
    \item 基于供需预测结果,大规模有序调动全城所有可用运力,实现资源最优化分配,
          力求解决正在发生的以及潜在供需失衡的状况,提升平台效率的同时最大化利用交通运力。
    \item I redesigned and built the new framework for transport capacity dispatch service, 4000 lines of Google-styled C++ code, thrift server.
 \end{itemize}

\datedsubsection{\textbf{司机端热力图}(负责人)}{Feb.\,2016 -- Jul.\,2016 (5 mos)}
 \begin{itemize}
    \item 基于对历史数据的统计并结合实时订单数据,给出当前全城范围内订单密集区域的分布,
          给司机提供有价值的听单位置参考,提高听单概率并减少司机空驶时间。
    \item 每天,热力图被近百万的司机,看近千万次。
    \item I added the function to apply configurable different filter on different cities, 
          and launched heatmap in 9 new cities: 
          深圳、武汉、天津、郑州、南京、
          重庆、长沙、大连、福州.
 \end{itemize}

\section{Selected Projects at Microsoft Research Asia}
\datedsubsection{\textbf{Document Rectification} Microsoft Research Asia}{Oct.\,2013 -- June.\,2014 (11 mos)}
 \begin{itemize}
    \item Rectify Curved Documents via Robust PCA (Principal Component Analysis) Framework.
    \item Reduce the running time from 100sec to 2sec, in 5,000 lines of C++11 code.
 \end{itemize}

\datedsubsection{\textbf{Click Through Rate Prediction} Microsoft Research Asia}{Aug.\,2013 -- Jun.\,2013 (3 mos)}
 \begin{itemize}
    \item Click Through Rate(CTR) Prediction of Bing Search via Logistic Regression(OWLQN) Algorithm.
    \item 2TB data, 2K lines of C++ code, boost, openmp, MPI.
 \end{itemize}

\datedsubsection{\textbf{Text Detection \& Recognization} Microsoft Research Asia}{Nov.\,2013 -- Jun.\,2014 (3 mos)}
 \begin{itemize}
    \item Extract textline from raw image.
    \item 1,000 lines of C++11 code, using opencv2.0 and tinyxml library.
 \end{itemize}

\datedsubsection{\textbf{Navigation Robot} Microsoft Research Asia}{May.\,2015 -- Jun.\,2015 (2 mos)}
 \begin{itemize}
    \item A navigation robot controlled by a smart phone.
    \item Add the feature to allow the server to dynamically push runnable code to the phone,
        controling its sensors and bluetooth modulo.
 \end{itemize}

\section{Selected Projects at Parttime}
\datedsubsection{\textbf{Wechat Sync Robot} }{Jun.\,2016 -- Jun.\,2016 (1 mos)}
 \begin{itemize}
    \item A robot that sync the messages between wechat groups for Microsoft Research Asia Alumni.
    \item It allows the 1000+ alumni to chat to each other as if we are in the same group.
    \item It is developed basing on some open source solutions, hacking the web wechat API by packet sniffer.
 \end{itemize}

 
% \section{\faInfo\ Miscellaneous}
% \begin{itemize}[parsep=0.5ex]
%   \item Blog: http://your.blog.me
%   \item GitHub: https://github.com/username
%   \item Languages: English - Fluent, Mandarin - Native speaker
% \end{itemize}

%% Reference
%\newpage
%\bibliographystyle{IEEETran}
%\bibliography{mycite}
\end{document}
